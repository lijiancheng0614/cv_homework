\documentclass[a4paper, 12pt, UTF8]{article}

\usepackage{xeCJK}
\setCJKmainfont[BoldFont={SimHei},ItalicFont={KaiTi}]{SimSun}

\usepackage{amsfonts}
\usepackage{amsmath}
\usepackage{graphicx}
\usepackage{indentfirst}
\usepackage{listings}
\lstset{
    columns=flexible,
    breakatwhitespace=false,
    breaklines=true,
    frame=single,
    numbers=left,
    numbersep=5pt,
    showspaces=false,
    showstringspaces=false,
    showtabs=false,
    stepnumber=1,
    rulecolor=\color{black},
    tabsize=2,
    texcl=true,
    escapeinside={\%*}{*)},
    extendedchars=false,
    mathescape=true,
}

\usepackage[colorlinks, citecolor=red]{hyperref}

\setlength{\evensidemargin}{-0.05in}
\setlength{\oddsidemargin}{-0.05in}
\setlength{\headheight}{-0.2in}
\setlength{\headsep}{0in}
\setlength{\textheight}{9.75in}
\setlength{\textwidth}{6.5in}
\setlength{\parindent}{2em}

\renewcommand{\baselinestretch}{1.5}

\begin{document}

\title{计算机视觉第2次作业}
\author{黎健成}
\date{2015210936}
\maketitle

% --------------------------------
\section{实验目的}

\begin{enumerate}

\item 熟悉3D特征。

\item 使用人体动作识别数据库进行实验。

\end{enumerate}


% --------------------------------
\section{实验要求}

\begin{enumerate}

\item 根据自身能力和兴趣情况,选择以下四种:

1、课件基本内容(基于光流及多特征的动作识别)

2、A compact optical flow based motion representation for real-time action recognition in surveillance scenes
(ICIP 2009)

3、Human action recognition in video via fused optical flow and moment features - Towards a hierarchical approach to complex scenario recognition
(MMM 2014)

4、Action recognition with improved trajectories
(ICCV 2013)

\item 采用课件中数据库,但不限,对比课件中算法精度,详细列出算法流程。

\item 鼓励采用新方法,以精度更高为目标。

\end{enumerate}


% --------------------------------
\section{实验环境}

操作系统:Ubuntu 14.04.3 LTS

开发环境:Python 2.7.6 + OpenCV 2.4.11

Python Library: Scikit-learn 0.17, numpy 1.10.2


% --------------------------------
\section{实验过程}

% ================================
\subsection{实验问题}

给定若干不同类别动作的视频,训练一个模型,预测未知动作类别的视频包含的动作。


% ================================
\subsection{实验解决方案}

实验解决过程按以下步骤进行:

\begin{enumerate}

\item 提取视频特征

\item 训练分类模型

\item 测试并计算准确率

\end{enumerate}

\subsubsection{提取视频特征}

这里使用HOF(Histograms of optical flow)特征,参考文献\cite{ref1},调用OpenCV中的\lstinline[language=Python]{cv2.calcOpticalFlowFarneback()}方法\textsuperscript{\cite{ref2}}——即使用Farneback方法获取每相邻两帧的光流信息;之后再计算直方图、方差等数据并合并成特征向量。

具体实现见\lstinline{hw2.py的__get_features()、video_features.py、optical_flow.py}。

\subsubsection{训练分类模型}

下载数据集,并根据网站信息把数据集分为训练集、验证集(可选)、测试集。

对训练集中每个视频提取特征向量,构成一个$m \times p$的矩阵,其中$m$表示视频的数量,$p$表示特征的长度。

调用Scikit-learn中的\lstinline[language=Python]{sklearn.multiclass.OneVsRestClassifier(sklearn.svm.LinearSVC())}方法\textsuperscript{\cite{ref3} \cite{ref4}}——即使用训练集的特征和标签训练一个一对多的SVM线性分类器。

具体实现见\lstinline{hw2.py的train()}。

\subsubsection{测试并计算准确率}

对测试集中每个视频提取特征向量,构成一个$n \times p$的矩阵,其中$n$表示视频的数量,$p$表示特征的长度。

使用训练时得到的分类器进行预测,并调用Scikit-learn中的\lstinline[language=Python]{sklearn.metrics.accuracy_score()}方法\textsuperscript{\cite{ref5}}计算准确率。

具体实现见\lstinline{hw2.py的test()}。


% ================================
\subsection{实验数据集}

\subsubsection{KTH数据集}

KTH数据集下载地址:\url{http://www.nada.kth.se/cvap/actions/}

KTH数据集有599段视频,包括6类动作:走(walking),慢跑(jogging),跑步(running),拳击(boxing),摇手(hand waving),鼓掌(hand clapping)。每类动作由25个不同的人分别在4个不同的场景(室外、室外放大、室外且穿不同颜色的衣服、室内)下完成。视频中背景相对静止,运动变化较小。

实验把数据集分为3个部分:训练集(8个人)、验证集(8个人)、测试集(9个人)。故训练集应有$8 \times 4 \times 6 = 192$个视频,验证集也有$192$个视频,测试集有$9 \times 4 \times 6 = 216$个视频,但其中训练集有一个视频(person13\_handclapping\_d3)缺失。

\subsubsection{Youtube数据集}

Youtube数据集下载地址:\url{http://crcv.ucf.edu/data/UCF_YouTube_Action.php}

Youtube数据集包括11类动作:篮球射球(basketball shooting),自行车(biking/cycling),划水(diving),挥高尔夫杆(golf swinging),马术表演(horse back riding),足球跑(soccer juggling),挥舞(swinging),挥网球拍(tennis swinging),跳蹦蹦床(trampoline jumping),排球扣球(volleyball spiking),遛狗行走(walking with a dog)。每类动作分为具有公共特征的25个组,每组。视频中背景较杂乱,运动变化较大。

\subsubsection{Hollywood2数据集}

Hollywood2数据集下载地址:\url{http://www.di.ens.fr/~laptev/actions/hollywood2/}

Hollywood2数据集包括12类动作:打电话(AnswerPhone),开车(DriveCar),吃饭(Eat),打架(FightPerson),下车(GetOutCar),握手(HandShake),拥抱(HugPerson),接吻(Kiss),跑步(Run),坐下(SitDown),仰卧起坐(SitUp),起立(StandUp)。视频由69个不同的好莱坞电影中截取而成。

% ================================
\subsection{实验结果}

\subsubsection{KTH数据集}

对测试集,准确率为$58.8\%$,表[\ref{table_kth}]列出了不同类别预测的结果。

可见,对部分类别(boxing, handwaving)准确率较高,而部分类别(running, jogging)则准确率较低。从视频内容分析,可知running与jogging等部分类别较易混淆,故不容易预测。从方法上来考虑,在计算HOF特征时直接使用整段视频来处理,存在背景混淆、无用帧等问题。

与当前许多方法相比,准确率较低,仍需改进。如考虑先提取人的bounding box再进行提取HOF特征等。

\begin{table}[h!]
    \centering
    \caption{KTH数据集不同类别预测结果}
    \label{table_kth}
    \begin{tabular}{cccc}
        类别 & 测试集中视频数量 & 预测正确数量 & 准确率 \\ \hline
        boxing       & 36  & 33  & 91.7\% \\
        handclapping & 36  & 20  & 55.6\% \\
        handwaving   & 36  & 22  & 61.1\% \\
        jogging      & 36  & 16  & 44.4\% \\
        running      & 36  & 15  & 41.7\% \\
        walking      & 36  & 21  & 58.3\% \\
        all          & 216 & 127 & 58.8\%
    \end{tabular}
\end{table}
 

\subsubsection{Youtube数据集}

这个数据集的数据量较大,暂时还未完成。

\subsubsection{Hollywood2数据集}

这个数据集的数据量较大,暂时还未完成。

% --------------------------------
\section{实验感想}

通过这次实验,熟悉了光流、动作识别等,对分类器的使用也有一定的了解,实现了简单的基于视频特征的动作识别,基本达到了实验目的。


% --------------------------------
\renewcommand{\refname}{参考}
\begin{thebibliography}{9}
\bibitem{ref1} Chaudhry R, Ravichandran A, Hager G, et al. Histograms of oriented optical flow and binet-cauchy kernels on nonlinear dynamical systems for the recognition of human actions[C]//Computer Vision and Pattern Recognition, 2009. CVPR 2009. IEEE Conference on. IEEE, 2009: 1932-1939.
\bibitem{ref2} Optical Flow. opencv dev team. 最后修订于2014年11月10日.  \url{http://docs.opencv.org/3.0-beta/doc/py_tutorials/py_video/py_lucas_kanade/py_lucas_kanade.html}
\bibitem{ref3} Multiclass and multilabel algorithms. scikit-learn developers. \url{http://scikit-learn.org/stable/modules/multiclass.html#one-vs-the-rest}
\bibitem{ref4} Support Vector Machines. scikit-learn developers. \url{http://scikit-learn.org/stable/modules/svm.html}
\bibitem{ref5} Model evaluation: quantifying the quality of predictions. scikit-learn developers. \url{http://scikit-learn.org/stable/modules/model_evaluation.html#classification-metrics}
\end{thebibliography}

\end{document}